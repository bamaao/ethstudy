\ifx\allfiles\undefined
\documentclass[UTF8]{ctexart}
\title{密码学安全}
\author{邹远春}
\date{}
\usepackage{xeCJK}
\usepackage{graphicx}
\usepackage{listings}
\usepackage{verbatim}
\begin{document}
\maketitle
\else
\chapter{密码学安全}
\fi
\section{公钥密码学}

\subsection{RSA}

\subsection{离散对数ElGamal}

\subsection{椭圆曲线密码学ECC}

\subsection{数字签名}

\subsubsection{Schnorr}

\subsubsection{Secp256k1}

[以太坊源代码分析] IV. 椭圆曲线密码学和以太坊中的椭圆曲线数字签名算法应用
https://blog.csdn.net/teaspring/article/details/77834360

\section{哈希}

\paragraph{SHA-3哈希加密}
SHA-3在2015年8月由美国标准技术协会(NIST)正式发布,作为Secure Hash Algorithm家族的最新一代标准。

\section{安全多方计算}

\subsection{保护隐私的电子投票协议}

\section{比特承诺}

\section{高级签名协议}

\subsection{盲签名}

\subsection{环签名}

\section{零知识证明}

\subsection{$\sum$协议}

\subsection{Bulletproofs}

\subsection{zkSnarks}

\subsection{zkStarks}

\section{秘密分享与门限密码学}

\subsection{密码分享}

\subsubsection{VSS}

\subsubsection{基于VSS的随机数生成}

\subsection{门限密码}

\subsection{门限签名}

\subsubsection{BLS}

\section{量子安全}

%\begin{multicols}{2}
%清朝中期, 外国人就开始中国采集植物标本。清朝中期, 外国人就开始中国采集植物标本。清朝中期, 外国人就开始中国采集植物标本。清朝中期, 外国人就开始中国采%集植物标本。清朝中期, 外国人就开始中国采集植物标本。清朝中期, 外国人就开始中国采集植物标本。
%\end{multicols}
\ifx\allfiles\undefined
\end{document}
\fi